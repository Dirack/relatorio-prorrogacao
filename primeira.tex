\chapter{PRIMEIRA ITERAÇÃO}
\label{cap4}

A primeira iteração do algoritmo de inversão tem por objetivo a obtenção do gradiente de
velocidades em profundidade do modelo de velocidades de background, este resultado é oque
chamamos de primeira iteração. O modelo inicial utilizado é um modelo de velocidades homogêneo
e de velocidade constante igual a 1.5Km/s correspondente a velocidade próxima da superfície de aquisição. 

FOTO MODELO BACKGROUND

Nesta etapa, é utilizado o algoritmo Very Fast Simulated Annealing (VFSA) para realizar a otimização do
gradiente de profundidade do modelo de velocidades. O modelo de velocidades é atualizado a cada iteração do
algoritmo para cada valor do gradiente e segue a seguinte função de velocidades:

\begin{equation}
\label{eq:4.1}
v(z)=z g_z+v_0
\end{equation}

Em \ref{eq:4.1} a velocidade $v(z)$ cresce linearmente com a profundidade.

FOTO MODELO GZ

O critério de otimização do gradiente de velocidades é 
TODO: critério de otimização semblance, tempos de trânsito e CRE



