\chapter{INTRODUÇÃO}
\label{cap1}

A estratégia utilizada para a inversão do modelo de velocidades apresentada na
qualificação de soutorado \cite{relatorio} foi abandonada em prol de uma nova estratégia de
inversão baseada na NIP tomografia e na stereo tomografia \cite{niptomo,stereo}.
A inversão do modelo de velocidades a partir da simulação e focalização
de difrações sobre a seção empilhada \cite{relatorio}  
possui uma séria limitação: Esta necessita do modelo de velocidades
de sobretempo normal (informação a priori) para simular as difrações sobre a seção empilhada e realizar a inversão através da focalização destas difrações. O empilhamento Elemento de Reflaxão Comum (ERC)
não produz o modelo de velocidades de sobretempo normal,
a seção empilhada é obtida a partir dos parâmetros $R_{NIP}$ e $\beta_0$ do método 
Superfície de Reflexão Comum (SRC) de afastamento nulo.

Assim, neste relatório, propomos uma nova estratégia para a inversão do modelo de velocidades em
profundidade utilizando um método baseado na NIP tomografia e na stereo tomografia: A partir da seção
empilhada ERC obtida \cite{relatorio} realizamos o picking dos tempos de trânsito sobre os eventos de
reflexão na seção. Estes tempos de trânsito são utilizados para localizar fontes pontuais sobre os relfetores
em um modelo de velocidades de background. Isto é feito através do traçamento de raios normais a partir
da superfície de aquisição em direção ao modelo em profundidade. A direção inicial do raio normal
é dada pelo ângulo de incidência $\beta_0$.
Realizamos a modelagem direta traçando raios de reflexão a partir destas fontes pontuais em profundidade
até atingirem a superfície de reflexão e comparamos os tempos de trânsito obtidos com os tempos de trânsito
calculados através da fórmula do ERC. O pressuposto da nossa estratégia de inversão é que os tempos
de trânsito obtidos com o traçamento de raios serão próximos aos tempos de trânsito calculados para
o modelo de velocidades ótimo.

Assim, utilizamos a soma das diferenças nos tempos de trânsito obtidos com o traçamento de raios
no modelo de velocidades em profundidade e os tempos de trânsito calculados pela fórmula do ERC como
critério de convergência da inversão. A atualização do modelo é realizada pelo algoritmo
Very Fast Simulated Annealing (VFSA) que atualiza as perturbações sobre o modelo de velocidades
de background definido sobre uma malha regular. Após a atualização, os raios de reflexão são
traçados novamente e as diferenças dos tempos de trânsito são recalculadas. Este processo é
repetido até o número máximo de iterações e o modelo otimizado é armazenado. O traçamento
dos raios normais e localização das fontes PIN é refeito utilizando este novo modelo de velocidades
e o processo é reiniciado. Após um número determinado de iterações, o modelo de velocidades otimizado 
e a correta localização das fontes PIN sobre o refletor são obtidos.

A estratégia de inversão descrita acima consiste no desenvolvimento da minha tese Doutorado
realizado durante as restrições do isolamento social durante a pandemia de CoVid-19. Estas
restrições impediram o andamento normal da minha pesquisa e são a justificativa para o meu
pedido de prorrogação do prazo de defesa por mais um ano. O relatório a seguir apresenta 
esta justificativa em detalhes, oque foi desenvolvido durante as restrições da pandemia
e oque eu pretendo realizar durante o prazo de prorrogação.



