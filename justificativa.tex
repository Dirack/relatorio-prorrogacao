\chapter{JUSTIFICATIVA}
\label{justificativa}

O meu pedido de prorrogação de bolsa de doutorado foi embasado nos artigos 2° e 4° da portaria 55
da COORDENAÇÃO DE APERFEIÇOAMENTO DE PESSOAL DE NÍVEL SUPERIOR - CAPES de 29 de Abril de 2020.
Esta portaria dispõe sobre a prorrogação excepcional
dos prazos de vigênciade das bolsas de mestrado e doutorado no país da CAPES
e estabelece os critérios necessários para o pedido de prorrogação do período de vigência das bolsas.
Em virtude da aceitação do meu pedido de prorrogação de bolsa de doutorado por parte da CAPES,
considero pertinente a utilização da mesma justificativa para o pedido de extensão do prazo da defesa da tese
por mais um ano além do período normal de defesa.

Como explicitado no Artigo 2° da Portaria 55 da CAPES abaixo:

"Art. 2º Fica autorizada, nos termos desta Portaria, a prorrogação dos prazos de vigência das bolsas de estudo de mestrado e doutorado concedidas no âmbito dos programas e acordos de competência da Diretoria de Programas e Bolsas no País da CAPES quando as restrições decorrentes do isolamento social necessário ao combate à pandemia da CoViD-19 tenham afetado o regular desenvolvimento do curso de pós-graduação ou o adequado desempenho dos mestrandos e doutorandos."

E explicitado no Art. 4°:

"Art. 4º São circunstâncias aptas a dar ensejo à prorrogação autorizada por esta Portaria:
I - o cancelamento ou o adiamento de atividades presenciais necessárias ao desenvolvimento do curso, que não possam ser supridas adequadamente por meio de ensino à distância ou outros meios, tais como atividades laboratoriais ou de campo, coleta de dados, entre outras; II - restrições temporárias de acesso a instalações necessárias ao desenvolvimento das atividades do curso;
III - outras situações que tenham imposto dificuldades não antevistas aos mestrandos e doutorandos, respeitados os limites fixados por esta Portaria."

O disposto nos artigos acima embasa a minha justificativa para o pedido de extensão do prazo de defesa da minha
tese de doutorado, pois durante o período de isolamento social me encontrei impossibilitado de desenvolver as atividades laboratoriais de que depende o desenvolvimento regular da minha tese de Doutorado. Haja vista que os experimentos computacionais que são a base dos resultados a serem apresentados na minha defesa não puderam ter a sua devida continuidade em virtude da drástica redução do poder computacional disponível para realizá-los. Cito alguns dos empecilhos abaixo:

\begin{itemize}


\item Os experimentos numéricos desenvolvidos no Laboratório de Inversão de Ondas Sísmicas (LIOS) dependem de estratégias de otimização e paralelização realizadas nas Virtual Central Processing Units (VCPUS) dos computadores do laboratório através do programa SCons e do pacote de processamento sísmico Madagascar.
Estes computadores possuem 12 VCPUS cada, 2 núcleos em cada um dos seus 6 processadores Xeon.

\item O Meu computador pessoal (notebook positivo stilo xci3650) dispõe de 2 núcleos do processador Intel Celeron. O ambiente do laboratório é refrigerado e utiliza a energia elétrica da UFPA. Um ambiente refrigerado é necessário para um maior desempenho dos experimentos numéricos realizados e estes podem rodar os algoritmos de inversão em looping por vários dias seguidos. O calor gerado reduz o desempenho da máquina e sua vida útil e o gasto com energia elétrica também deve ser considerado. Além disso, os computadores e o ambiente do laboratório são fabricados para este tipo de desenvolvimento de
alto desempenho e o meu notebook não é.

\item A atividade no laboratório permite realizar o desenvolvimento e a experimentação ao mesmo tempo. No LIOS eu realizava o desenvolvimento dos algoritmos e a escrita do código no meu notebook enquanto rodava os experimentos numéricos anteriormente citados nos computadores do laboratório, oque já não posso mais fazer.

\item Os computadores do LIOS permitem uma estratégia simulando a "clusterização". Estes computadores me permitiam rodar vários experimentos numéricos e testes de diferentes modelos sintéticos ao mesmo tempo, em diferentes máquinas, e com a estratégia de paralelização supracitada nas VCPUS de cada máquina (cada uma com 12 núcleos), controlando e balanceando a carga com o software de integração contínua "Jenkins" e acessando as máquinas via OpenSSH.
Oque não pode mais ser feito e custaria milhares de reais para reproduzir em algum serviço de VPS ou Cloud Computing.

\item Para que eu possa realizar as atividades citadas acima, necessárias ao desenvolvimento da minha Tese de doutorado, irei necessitar tanto do tempo extra quanto do recurso financeiro da minha bolsa de doutorado, como me dedico exclusivamente à pesquisa, esta representa a minha única fonte de renda.

\end{itemize}

A justificativa apresentada acima é suficiente e está contemplada pelos artigos supracitados da portaria 55. Esta justificativa é evidenciada pela lista de empecilhos apresentados. Estes contribuem para reduzir o desempenho do regular desenvolvimento da minha Tese de Doutorado pelos meses que durarem o isolamento social. Por conseguinte, como esta
justificativa foi considerada pertinente por parte da CAPES para a prorrogação do período de vigência da minha bolsa
de doutorado, considero também justificado o meu pedidido de prorrogação do prazo para a defesa da tese de doutorado por mais
um ano além do prazo normal de defesa.
